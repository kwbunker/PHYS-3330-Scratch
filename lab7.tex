%%%%%%%%%%%%%%%%%%%%%%%%%%%%%%%%%%%%%%%%%
% CU Boulder Physics Lab Writeup One Column
% LaTeX Template
% Version 1.0 (2022-10-05)
%
% This template has been downloaded from:
% http://www.LaTeXTemplates.com
%
% Original author:
% Mathias Legrand (legrand.mathias@gmail.com) titled Stylish Article 
% With extensive modifications by:
% Vel (vel@latextemplates.com)
% Further modifications for CU Boulder Physics by:
% Kristopher Bunker (kristopher.bunker@colorado.edu)
%
% License:
% CC BY-NC-SA 3.0 (http://creativecommons.org/licenses/by-nc-sa/3.0/)
%
%%%%%%%%%%%%%%%%%%%%%%%%%%%%%%%%%%%%%%%%%

%----------------------------------------------------------------------------------------
%	PACKAGES AND OTHER DOCUMENT CONFIGURATIONS
%----------------------------------------------------------------------------------------

\documentclass[10pt]{PhysLab1C} % Document font size

\usepackage[english]{babel} % Specify a different language here - english by default

%----------------------------------------------------------------------------------------
%	COLUMNS
%----------------------------------------------------------------------------------------

\setlength{\columnsep}{0.55cm} % Distance between the two columns of text
\setlength{\fboxrule}{0.75pt} % Width of the border around the abstract

%----------------------------------------------------------------------------------------
%	COLORS
%----------------------------------------------------------------------------------------

\definecolor{color1}{RGB}{0,0,90} % Color of the article title and sections
\definecolor{color2}{RGB}{0,20,20} % Color of the boxes behind the abstract and headings

%----------------------------------------------------------------------------------------
%	HYPERLINKS
%----------------------------------------------------------------------------------------

\usepackage{hyperref} % Required for hyperlinks

\hypersetup{
	hidelinks,
	colorlinks,
	breaklinks=true,
	urlcolor=color2,
	citecolor=color1,
	linkcolor=color1,
	bookmarksopen=false,
	pdftitle={Title},
	pdfauthor={Author},
}

%----------------------------------------------------------------------------------------
%	LAB AND COURSE INFORMATION
%----------------------------------------------------------------------------------------

\CourseInfo{Electronics for the Physical Sciences \vert ~ \textbf{PHYS 3330}} %
\Department{\copyright \ Department of Physics \vert ~ \textbf{University of Colorado Boulder} \ \vert ~ \textbf{\today}} %
\Copyright{\today} %
\LabTitle{Bipolar Junction Transistors} % Lab Title

%----------------------------------------------------------------------------------------
%	ABSTRACT
%----------------------------------------------------------------------------------------

\Abstract{\textbf{Lab 7:} Transistor Amplifiers}

%----------------------------------------------------------------------------------------

\begin{document}

\maketitle % Output the title and abstract box

%\tableofcontents % Output the contents section

\thispagestyle{firstpage} % Removes page numbering from the first page

%----------------------------------------------------------------------------------------
%	ARTICLE CONTENTS
%----------------------------------------------------------------------------------------

\section{Goals}
In this lab, you will begin learning how transistors work by creating a dual-stage amplifier out of a common
emitter amplifier plus an emitter follower and use this as an audio amplifier.

Proficiency with new equipment:

\begin{itemize}
\item
  Bipolar junction transistors
\end{itemize}

Modeling the physical system:

\begin{itemize}
\item
  Develop a model of bipolar junction transistors
\end{itemize}

Applications:

\begin{itemize}
\item
  Build a dual-stage amplifier
\item
 Use the amplifier to drive a speaker
\end{itemize}


%------------------------------------------------

\section{Definitions}

$\mathbf{h_{fe}}$ \textbf{or} $\bm{\beta}$ - transistor current gain intrinsic to the transistor itself.

$\mathbf{V_{E},~V_{B},~V_{C}}$ - voltages at the emitter, base, and collector, respectively. These are the quiescent (DC) voltages.

$\mathbf{v_{in},~v_{out}}$ - input and output voltages; written in lower-case to indicate an AC signal only (not including any DC
offset).

\textbf{Blocking capacitor} or \textbf{coupling capacitor} - capacitors used to transmit an AC signal while blocking the DC
component.


%------------------------------------------------

\section{Bipolar Junction Transistors - General Use}

An electrical signal can be amplified using a device that allows a small current or voltage to control the flow of
a much larger current from a DC power source. Transistors are the basic devices providing control of this kind.
There are two general types of transistors, bipolar and field-effect. The difference between these two types is
that for bipolar devices an input current controls the large current flow through the device, while for field-effect transistors an input voltage provides the current control. In this experiment we will build a two-stage
amplifier using two bipolar transistors.

In many practical applications it is better to use an op-amp as a source of gain rather than to build an amplifier
from discrete transistors. A good understanding of transistor fundamentals is nevertheless essential because
op-amps are built from transistors. We will learn later about digital circuits, which are also made from
transistors. In addition to the importance of transistors as components of op-amps, digital circuits, and an
enormous variety of other integrated circuits, single transistors (usually called “discrete” transistors) are used
in many applications. They are important as interface devices between integrated circuits and sensors,
indicators, and other devices used to communicate with the outside world. High-performance amplifiers
operating from DC through microwave frequencies use discrete transistor “front-ends” to achieve the lowest
possible noise. Discrete transistors are generally much faster than op-amps. The device we will use this week
has a gain-bandwidth product of 300 MHz (compared to 5 MHz for the LF356 op-amp you have been using).

The three terminals of a bipolar transistor are called the emitter, base, and collector (Figure \ref{NPN}). A small current
into the base controls a large current flow from the collector to the emitter. The current at the base is typically
about 1\% of the collector-emitter current. This means that the transistor acts as a current amplifier with a typical current gain ($h_{fe}$ or $\beta$) of about 100. Moreover, the large collector current flow is almost independent of the
voltage across the transistor from collector to emitter. This makes it possible to obtain a large amplification of
voltage by having the collector current flow through a resistor.

\begin{figure}
    \centering
    \includegraphics{tmp.png}
    \caption{Diagram of an NPN bipolar junction transistor (left) and schematic symbol (right)}
    \label{NPN}
\end{figure}

%------------------------------------------------

\section{Current Amplifier Model of a Bipolar Transistor}

From the simplest point of view a bipolar transistor is a current amplifier. The current flowing from collector to
emitter is equal to the base current multiplied by a factor. An NPN transistor like the 2N3904 operates with the
collector voltage at least a few tenths of a volt above the emitter, and with a current flowing into the base.
There are also PNP transistors with opposite polarity voltages and currents. The base-emitter junction acts like
a forward-biased diode with a 0.6 V drop:

$$V_E \approx V_B-0.6~V$$

Under these conditions, the collector current is proportional to the base current:

$$I_C = h_{fe}I_B$$

The constant of proportionality (‘current gain’) is called $h_{fe}$ because it is one of the "h-parameters", which are a
set of numbers that give a complete description of the small-signal properties of a transistor. It is important to
keep in mind that hfe is not really a constant. It depends on collector current, and it can vary by 50\% or more
from device to device.

If you want to know the emitter current, rather than the collector current, you can find it by current
conservation:

$$I_E = I_B+I_C = I_C\left(\frac{1}{h_{fe}+1}\right)$$

The difference between $I_C$ and $I_E$ is almost never important since $h_{fe}$ is normally in the range of 100–1000, so
you can generally assume that $I_E$ = $I_C$. Another way to say this is that the base current is very small compared
to the collector and emitter currents.

%------------------------------------------------

\section{Emitter Follower and Common Emitter Amplifier}

\begin{figure}[ht]
\centering
\begin{subfigure}[t]{0.3\textwidth}
\centering
\begin{circuitikz}[american voltages]
    \draw (0,0) node[npn](Q){};
    \draw (Q.B) to [short, -o] ++(-1,0) node[anchor = east]{$V_{in}$};
    \draw (Q.E) to [R, l=$R_E$] ++(0,-2) node[ground]{};
    \draw (Q.E) to [short,*-o] ++(1.5,0) node[anchor = west]{$V_{out}$};
    \draw (Q.C) to [short, -o] ++(0,2) node[anchor = south]{$+15~V$};
\end{circuitikz}
 \caption{Emitter follower}
 \label{emitterf}
 \end{subfigure}
 \hspace{2cm}%
 \begin{subfigure}[t]{0.3\textwidth}
\centering
\begin{circuitikz}[american voltages]
    \draw (0,0) node[npn](Q){};
    \draw (Q.B) to [short, -o] ++(-1,0) node[anchor = east]{$V_{in}$};
    \draw (Q.E) to [R, l=$R_E$] ++(0,-2) node[ground]{};
    \draw (Q.C) to [short,*-o] ++(1.5,0) node[anchor = west]{$V_{out}$};
    \draw (Q.C) to [R, l=$R_C$] ++(0,2) to [short, -o] ++(0,0) node[anchor = south]{$+15~V$};
\end{circuitikz}
 \caption{Common emitter amplifier}
 \label{emittera}
 \end{subfigure}
 \hspace{0.75cm}%
  \medskip
 \caption{Two basic transistor circuits}
 \label{emitter}
 \end{figure}

We will begin by constructing a common emitter amplifier, which operates on the principle of a current
amplifier. However, a major fault of the common emitter amplifier is its high output impedance. This problem
can be fixed by adding a second circuit, the emitter follower, as a second stage. The common-emitter
amplifier and the emitter follower are the most common bipolar transistor circuits.

First consider the emitter follower (Figure \ref{emitterf}). The output (emitter) voltage is always 0.6 V (one diode drop)
below the input (base) voltage. A small AC signal of amplitude $v_{in}$ at the input will therefore give a signal $v_{in}$ at
the output, i.e. the output just “follows” the input. There is no gain (the gain is 1) but we will see later that this
circuit is still useful because it has high input impedance and low output impedance.

Now consider the common emitter amplifier (Figure \ref{emittera}). A small AC signal of amplitude $v_{in}$ at the input will
give the same AC signal $v_{in}$ at the emitter (just like the emitter follower). This will cause a varying current of
amplitude $v_{in}/R_E$ to flow from the emitter to ground. As we found above, $I_E\gg I_C$, and so the same current will
flow through $R_C$. This current generates $v_{out} = –R_C(v_{in}/R_E)$. Thus, the common emitter stage has a small-signal
AC voltage gain of:

$$G=\frac{-R_C}{R_E}$$

\textbf{Note:} this gain is about the change in voltage, which is a bit different than the gains you have been working
with so far.


%------------------------------------------------

\section{Biasing a Transistor Amplifier}

Figure \ref{emittera2} shows the basic common emitter amplifier, while Figure \ref{emitterabn} shows the common emitter
amplifier circuit that you will build with the biasing resistors, power filter capacitor, and coupling capacitors in
place. The power filter capacitor connecting +15 V to ground serves to ensure that electronic noise from the
power supply does not enter the circuit. The coupling capacitors on the input and output serve to transmit the
AC signal and block the DC voltage. For the input, this is essential because if you feed DC voltage into
something providing its own voltage, bad things will happen. In this case, if the biasing DC voltage is directly
connected to the output of the function generator ($v_{in}$) the output buffer in the function generator will break.
For the output, the coupling capacitor is simply convenient as we don’t care about the DC voltages, just the AC
signal.



\begin{figure}[ht]
\centering
\begin{subfigure}[t]{0.3\textwidth}
\centering
\begin{circuitikz}[american voltages]
    \draw (0,0) node[npn](Q){};
    \draw (Q.B) to [short, -o] ++(-1,0) node[anchor = east]{$V_{in}$};
    \draw (Q.E) to [R, l=$R_E$] ++(0,-2) node[ground]{};
    \draw (Q.C) to [short,*-o] ++(1.5,0) node[anchor = west]{$V_{out}$};
    \draw (Q.C) to [R, l=$R_C$] ++(0,2) to [short, -o] ++(0,0) node[anchor = south]{$+15~V$};
\end{circuitikz}
 \caption{Common emitter amplifier}
 \label{emittera2}
 \end{subfigure}
 \hspace{1cm}%
 \begin{subfigure}[t]{0.3\textwidth}
\centering
\begin{circuitikz}[american voltages]
    \draw (0,0) node[npn](Q){2N3904};
    \draw (Q.B) -- ++(-2,0) to [C, l2=$C_{in}$  and 220 nF, l2 halign=c] ++(-1,0) 
        to [short, -o] ++(-1,0) node[anchor = east]{$V_{in}$};
    \draw (Q.E) to [R, l2=$R_E$ and 1 k$\Omega$] ++(0,-3) to ++(-1,0) node[ground]{}
        to ++(-1,0) to [R, l2=$R_2$ and 10 k$\Omega$, l2 halign=c] ++(0,3.77)
        to [short, *-] ++(0,1.25) to [R, l2=$R_1$ and 47 k$\Omega$, l2 halign=c] ++(0,2) to [short, -*] ++(0,.52);
    \draw (Q.C) to [short,*-] ++(1.5,0) to [C, l2_=$C_{out}$  and 47 $\mu $F, l2 halign=c] ++(1,0) 
        to [short,-o] ++(1,0) node[anchor = west]{$V_{out}$};
    \draw (Q.C) to [R, l2_=$R_C$ and 2.74 k$\Omega$, l2 halign=c] ++(0,3) coordinate(FT) 
        to [short, *-o] ++(1,0) node[anchor = west]{$+15~V$}
        (FT) to ++(-3,0) to [C, l2=$C$  and 47 $\mu$F, l2 halign=c] ++(-2,0) node[ground]{};
\end{circuitikz}
\captionsetup{%
    justification=raggedleft, width=6cm%
}
 \caption{Common emitter amplifier with biasing network and coupling capacitors in place}
 \label{emitterabn}
 \end{subfigure}
 \hspace{4cm}%
  \medskip
 \caption{Emitter amplifier circuits circuits}
 \label{emitter2}
 \end{figure}

This brings us to the biasing network. While we are usually trying to amplify a small AC signal, it is essential
that we setup the proper “quiescent point” or bias voltages. These are the DC voltages present when the
signal is zero. Setting up the proper biasing often requires an iterative process. Here are the steps to
understand what is happening in a transistor setup.



\begin{enumerate}

    \item 
     Set the DC voltage of the base, $V_B$, with a voltage divider ($R_1$ and $R_2$ in Figure \ref{emitterabn}).

    \item 
    The emitter voltage, $V_E$, will be 0.6 V less than the base voltage.

    \item 
    Between the emitter and ground, there is one resistor so we can determine the current flowing
through that resistor, which is the same as the current flowing out of the emitter: $I_E = V_E/R_E$.
    \item 
    Recall that the emitter and collector current are approximately equal so $I_C = I_E$.
    \item 
    Using this current, we can calculate voltage drop across the collector resistor and subtract that from
the power supply voltage, $V_{PS}$ = +15 V, to get the collector voltage: $V_C = V_{PS}~ –~ I_CR_C$.
    \item 
    If we have capacitors or inductors in parallel with (or in place of) the resistors $R_C$ and $R_E$, then we would
use the AC impedances.
\end{enumerate}

For an emitter follower, the biasing is about the same except that the collector is usually tied to the positive
supply voltage so $V_C = V_{PS}$.

Determining the best values to use for the resistors and capacitors depends on what the circuit needs to do.

%------------------------------------------------

\section{Output Range of the Common Emitter Amplifier (Clipping Voltages)}

Even with proper biasing of the transistor, the output voltage has a range that is less than 0-15 V. Let’s
determine the maximum and minimum output voltages. Since $V_{out} = V_{PS}~ –~ I_CR_C$, the maximum voltage will
occur when $I_C = 0$ and the minimum voltage when $I_C$ is a maximum.

\textbf{Maximum voltage:} This occurs when the transistor is turned off and no current is flowing. As there is no
current flowing through $R_C$, there is no voltage drop across it. Thus $V_{out} = V_{max} = V_{PS} = 15 ~V$.

\textbf{Minimum Voltage:} This occurs when the transistor is fully on. The maximum current is flowing and there is
very little voltage drop across the transistor. The voltage is:

$$V_{out} = V_{min} = V_{PS} ~– ~R_C I_C~ (max)$$

$I_C~(max)$ can be found by considering the voltage drop from the power supply to ground when there is no
voltage drop across the transistor: $V_{PS}~-~R_C I_C ~-~ R_E I_E = 0$.

%------------------------------------------------

\section{Input and Output Impedances of the Common Emitter Amplifier}

The input impedance is the same for both the emitter follower and the common emitter amplifier. The input
impedance looking into the base of the common emitter amplifier is

$$r_{in} = R_E h_{fe}$$

where $R_E$ is whatever impedance is connected to the emitter. If there is no load attached, it is just the emitter
resistor. However, if there is a load attached, $R_E$ will be the emitter resistor in parallel with the input
impedance of the load (or the next stage). This input impedance is for the base of the transistor. If there is a
biasing network in place, then the input impedance of the \textit{circuit} will be $r_{in}$ in parallel with the base bias resistors.

The output impedance of the common emitter amplifier (Figure 2b) is just equal to the collector resistor $R_C$:

$$r_{out} = R_C ~~\text{(common emitter)}$$

The output impedance of the emitter follower is found to be:

$$r_{out} = \frac{R_B}{h_{fe} +1}~~ \text{(emitter follower)}$$

where $R_B$ indicates whatever impedance is connected to the base. To be more precise, one should also include
the emitter resistor in parallel with rout for the true output impedance of the circuit, but this is usually not
necessary as rout is usually much smaller than $R_E$. Also, please note the next section gives a more precise
estimate of $r_{out}$.

%------------------------------------------------

\section{Ebers-Moll Model of a Bipolar Transistor}

Instead of using the current amplifier model, one can take the view that the collector current $I_C$ is controlled by
the base-emitter voltage $V_{BE}$. For our purposes, the Ebers-Moll model modifies our current amplifier model of
the transistor in only one important way. For small variations about the quiescent point, the transistor now
acts as if it has a small internal resistor, $r_e$, in series with the emitter. The magnitude of the intrinsic emitter
resistance, $r_e$, depends on the collector current $I_C$:

$$r_e = 25~\Omega \left(\frac{1~mA}{I_C}\right)$$

The presence of the intrinsic emitter resistance, $r_e$, modifies the above input and output impedances to:
 
$$r_{in} = (R_e + r_e)~h_{fe}~~ \text{(common emitter amplifier \textit{and} emitter follower)}$$
 
$$r_{out} = \frac{R_B}{h_{fe}+1}+r_e~~ \text{(emitter follower)}$$
 
In addition, this modifies the gain of the common emitter amplifier to:

$$G = \frac{-R_C}{R_E+r_e}$$

which shows that the common emitter gain is not infinite when the external emitter resistor goes to zero.

%------------------------------------------------

\section{Useful Readings}

%------------------------------------------------

\section{Prelab}

%------------------------------------------------

\section{Common Emitter Amplifier}

%------------------------------------------------

\section{Dual Stage Amplifier}

%------------------------------------------------

\section{Summary and Conclusions}

Write a two-paragraph summary in your lab notebook of what you learned
and any important takeaways.

%------------------------------------------------

\section*{Appendix A: Title}

\addcontentsline{toc}{section}{Appendix A: Title} % Adds this section to the table of contents

%------------------------------------------------

\end{document}